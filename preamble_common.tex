% Add document information
\date{\today}
\author{Moritz Wachsmuth-Melm}

% Fix too many alphabets error
\newcommand\hmmax{0}
\newcommand\bmmax{0}

%====================================================%
%================== Basic packages ==================%
% Packages that just should be there in any preamble %
%====================================================%
\RequirePackage[T1]{fontenc} 
\RequirePackage[ngerman,english]{babel}
% \RequirePackage[outputdir=output]{minted} % Code syntax highlighting, has to be loaded before csquotes
\RequirePackage[final,nopatch=footnote]{microtype} % Better font spacing and stuff. Final option provides that it is not disabled in draft mode
\RequirePackage[autostyle]{csquotes} % Better quoting, but mainly here because it is recommended for babel and biblatex
\RequirePackage{graphicx}

%=========%
% SIUnitx %
% =========%
\RequirePackage[
per-mode=reciprocal-positive-first, % Format of units (fraction or reciprocal)
locale=US, % Set locale
range-units=single, % Don't repeat units in number ranges
list-units=single, % Don't repeat units in number lists
detect-all, % Apply same font settings to the number/unit as the surrounding text
forbid-literal-units=true, % Only allow predefined units to enforce consistency
use-xspace=true, % Correct space after unit
]{siunitx} % Nice SI units, but also for number formatting/alignment in tables (with s and S column specifier)
% Declare custom units (mainly chemistry and biology related)
\DeclareSIUnit{\basepair}{bp} % Basepairs
\DeclareSIUnit{\base}{b} % Bases
\DeclareSIUnit{\molar}{\textsc{m}} % Molarity (mol/l)
\DeclareSIUnit{\mmolar}{\milli\molar} % Short form
\DeclareSIUnit{\umolar}{\micro\molar} % Short form
\DeclareSIUnit{\enzymeunit}{U} % Enzyme activity units
\DeclareSIUnit{\electron}{e\textsuperscript{-}} % Electrons
\DeclareSIUnit{\xg}{xg} % Centrifugal force (g)
\DeclareSIUnit{\rpm}{rpm} % Rounds per minute (centrifugation)
\DeclareSIUnit{\hpi}{hpi} % Hours past infection (virology)
\DeclareSIUnit{\PFU}{\textsc{PFU}} % Plaque forming units (virology)
\DeclareSIUnit{\Da}{Da} % Plaque forming units (virology)
\DeclareSIUnit{\kDa}{kDa} % Plaque forming units (virology)
\DeclareSIUnit[number-unit-product = {}]{\percent}{\%} % Remove space before percent sign


%===========%
% Chemistry %
%===========%
\RequirePackage{chemformula} % Stuff like \ch{MgCl2}


%=============%
% Table stuff %
%=============%
\RequirePackage{booktabs} % \toprule, \midrule, \bottomrule, nicer tables (must have!)
\RequirePackage{multirow} % \multicolumn{num_of_columns}{X}{Content} and \multirow{num_of_rows}{X}{Content}


%=========================%
% Floats, figures, images %
%=========================%
\RequirePackage[
    inkscapearea=page,
    inkscapeformat=png,
    inkscapedpi=600,
    % inkscapelatex=false
]{svg}
\RequirePackage{tikz} % Figures you can draw on (text, lines, braces, stuff)
\usetikzlibrary{positioning}
\usetikzlibrary{calc}
\tikzset{
    tight/.style={inner sep=0, outer sep=0},
	tile/.style={tight,draw=none,fill=none,align=center}}

%=============%
% Tikz things %
%=============%
\newcommand{\helplines}{\draw[help lines] (0, 0) grid[xstep=0.1, ystep=0.1] (1, 1)}

% \scalebar{node}{length}{text}
% Length can be either an actual length (e.g. .5\textwidth)
% or a fraction of the node's width (e.g. 0.1).
% In the latter case, the node's inner and outer sep must be 0, otherwise the scalebar is wrong!
\newlength{\scalebarwidth}
\setlength{\scalebarwidth}{1ex}
\newcommand{\scalebar}[3]{
    \begin{tikzonnode}{#1}
        \draw[line width=\scalebarwidth,draw=white]
   		    ($(1, 0) + (-.25\scalebarwidth, 0.75\scalebarwidth)$) -- +(-#2, 0)
		    node[midway, font=\tiny\bfseries, text=black] {#3};
    \end{tikzonnode}
}

% Drawing on figures
\newenvironment{tikzonnode}[1]{%
    \begin{scope}[shift={(#1.south west)},x={(#1.south east)},y={(#1.north west)}]%
        }{%
    \end{scope}%
}
\newenvironment{tikzongraphics}[2][width=\textwidth]{
    \begin{tikzpicture}
        \node[align=center, anchor=south west,inner sep=0] (image) at (0,0) {\includegraphics[#1]{#2}};
        \begin{tikzonnode}{image}
        }{
        \end{tikzonnode}
    \end{tikzpicture}
}

\newcommand{\tikzoverlay}[1]{
    \tikz[overlay,remember picture] {
        #1
    }
}
\endinput
